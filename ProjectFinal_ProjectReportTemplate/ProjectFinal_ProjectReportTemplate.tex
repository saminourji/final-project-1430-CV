%%%%%%%%%%%%%%%%%%%%%%%%%%%%%%%%%%%%%%%%%%%%%%%%%%%%%%%%%%%%%%%%%%%%%%%%%%%%%%%%%%%%%%%%%%%%%%%%
%
% CSCI 1430 Written Question Template
%
% This is a LaTeX document. LaTeX is a markup language for producing documents.
% Your task is to answer the questions by filling out this document, then to
% compile this into a PDF document.
%
% TO COMPILE:
% > pdflatex thisfile.tex

% If you do not have LaTeX, your options are:
% - VSCode extension: https://marketplace.visualstudio.com/items?itemName=James-Yu.latex-workshop
% - Online Tool: https://www.overleaf.com/ - most LaTeX packages are pre-installed here (e.g., \usepackage{}).
% - Personal laptops (all common OS): http://www.latex-project.org/get/ 
%
% If you need help with LaTeX, please come to office hours.
% Or, there is plenty of help online:
% https://en.wikibooks.org/wiki/LaTeX
%
% Good luck!
% The CSCI 1430 staff
%
%%%%%%%%%%%%%%%%%%%%%%%%%%%%%%%%%%%%%%%%%%%%%%%%%%%%%%%%%%%%%%%%%%%%%%%%%%%%%%%%%%%%%%%%%%%%%%%%
%
% How to include two graphics on the same line:
% 
% \includegraphics[width=0.49\linewidth]{yourgraphic1.png}
% \includegraphics[width=0.49\linewidth]{yourgraphic2.png}
%
% How to include equations:
%
% \begin{equation}
% y = mx+c
% \end{equation}
% 
%%%%%%%%%%%%%%%%%%%%%%%%%%%%%%%%%%%%%%%%%%%%%%%%%%%%%%%%%%%%%%%%%%%%%%%%%%%%%%%%%%%%%%%%%%%%%%%%

\documentclass[10pt,twocolumn,letterpaper]{article}
 
\usepackage{cvpr}
\usepackage{times}
\usepackage{epsfig}
\usepackage{graphicx}
\usepackage{amsmath}
\usepackage{amssymb}
\usepackage{booktabs}
\usepackage{microtype}
% From https://ctan.org/pkg/matlab-prettifier
\usepackage[numbered,framed]{matlab-prettifier}

\frenchspacing

% Include other packages here, before hyperref.

% If you comment hyperref and then uncomment it, you should delete
% egpaper.aux before re-running latex.  (Or just hit 'q' on the first latex
% run, let it finish, and you should be clear).
\usepackage[pagebackref=true,breaklinks=true,letterpaper=true,colorlinks,bookmarks=false]{hyperref}

\cvprfinalcopy
\def\cvprPaperID{****}
\def\httilde{\mbox{\tt\raisebox{-.5ex}{\symbol{126}}}}
\ifcvprfinal\pagestyle{empty}\fi

\begin{document}

%%%%%%%%% TITLE
\title{CSCI 1430 Final Project Report\\Data Augmented AI Generated Detector}

% Make this document not anonymous
\author{
Everest Yang, Sami Nourji, Sujith Pakala, Tanay Subramanian\\
    \emph{TA:} Winston Li \\
    Brown University\\
}

\maketitle
%\thispagestyle{empty}

%%%%%%%%% ABSTRACT
\begin{abstract}
This document is a template for your final project reports, presented in a conference-paper style. It is sightly-more complicated LaTeX, but not much more complex than the earlier project reports. 
This document, along with your code, any supplemental material, and your 2-minute presentation, are qualitatively what determines your grade. 
\end{abstract}

%%%%%%%%% BODY TEXT
\section{Introduction}

Currently, there are issues related to digital misinformation and privacy. As chatbots and generative AI become more sophisticated, they are able to create hyper-realistic fake media. This has significant implications for politics, social trust, and personal security. Deepfakes have already been involved in election interference, celebrity impersonations, and malicious pranks. In the context of media and news, this project becomes relevant for the authenticity of digital content in an era where images and videos can be fabricated easily.
The difficulty lies in the fact that AI-generated content can replicate  details like lighting, shadows, and facial expressions with astonishing precision, making traditional detection methods less effective. Manual verification is not feasible at scale which is why automated tools capable of detecting fakes are necessary.

In this paper, we propose a Data Augmented AI Generator capable of determining real images vs AI-generated. We want to test the hypothesis that AI-generated images have particular features that allow humans to differentiate them from real ones. For example, the smooth texture that seems to appear in AI-generated images. By leveraging computer vision and deep learning techniques, including neural networks trained on diverse datasets of real and synthetic media, this project aims to identify these distinguishing features. The use of data augmentation will ensure the model's robustness across various scenarios, enabling it to detect AI-generated content with greater accuracy.

\section{Related Work}

Cite and discuss work that you used in your project, including any software used. Citations are written into a .bib file in BibTeX format, and can be called like this: Alpher et al.~\cite{Alpher04}. Here's a brief intro: \href{http://www.andy-roberts.net/writing/latex/bibliographies}{webpage}. \emph{Hint:} \$$>$ pdflatex \%docu, bibtex \%docu, pdflatex \%docu, pdflatex \%docu


\section{Method}

Describe the problem in a compact way. What was your approach to solving it? Include diagrams to help understanding. For instance, if you used a CNN, what was the architecture? Include equations as necessary, e.g., Pythagoras' theorem (Eq.~\ref{eq:example}):
\begin{equation}
x^2 + y^2 = z^2,
\label{eq:example}
\end{equation}
where $x$ is the the `adjacent edge' of a right-angled triangle, $y$ is the `opposite edge' of a right-angled triangle, and $z$ is the hypotenuse.

My code snippet highlights an interesting point.
\begin{lstlisting}[style=Matlab-editor]
one = 1;
two = one + one;
if two != 2
    disp( 'This computer is broken.' );
end
\end{lstlisting}

Lorem ipsum dolor sit amet, consectetur adipiscing elit, sed do eiusmod tempor incididunt ut labore et dolore magna aliqua. Ut enim ad minim veniam, quis nostrud exercitation ullamco laboris nisi ut aliquip ex ea commodo consequat. Duis aute irure dolor in reprehenderit in voluptate velit esse cillum dolore eu fugiat nulla pariatur. Excepteur sint occaecat cupidatat non proident, sunt in culpa qui officia deserunt mollit anim id est laborum.

Lorem ipsum dolor sit amet, consectetur adipiscing elit, sed do eiusmod tempor incididunt ut labore et dolore magna aliqua. Ut enim ad minim veniam, quis nostrud exercitation ullamco laboris nisi ut aliquip ex ea commodo consequat. Duis aute irure dolor in reprehenderit in voluptate velit esse cillum dolore eu fugiat nulla pariatur. Excepteur sint occaecat cupidatat non proident, sunt in culpa qui officia deserunt mollit anim id est laborum.

Lorem ipsum dolor sit amet, consectetur adipiscing elit, sed do eiusmod tempor incididunt ut labore et dolore magna aliqua. Ut enim ad minim veniam, quis nostrud exercitation ullamco laboris nisi ut aliquip ex ea commodo consequat. Duis aute irure dolor in reprehenderit in voluptate velit esse cillum dolore eu fugiat nulla pariatur. Excepteur sint occaecat cupidatat non proident, sunt in culpa qui officia deserunt mollit anim id est laborum.

\section{Results}

Present the results of the changes. Include code snippets (just interesting things), figures (Figures \ref{fig:result1} and \ref{fig:result2}), and tables (Table \ref{tab:example}). Assess computational performance, accuracy performance, etc. Further, feel free to show screenshots, images; videos will have to be uploaded separately to Gradescope in a zip. Use whatever you need.

\begin{table}
\begin{center}
\begin{tabular}{ l c }
\toprule
Method & Frobnability \\
\midrule
Theirs & Frumpy \\
Yours & Frobbly \\
Ours & Makes one's heart Frob\\
\bottomrule
\end{tabular}
\end{center}
\caption{Results. Please write an explanatory caption that makes the table/figure self-contained.}
\label{tab:example}
\end{table}

Lorem ipsum dolor sit amet, consectetur adipiscing elit, sed do eiusmod tempor incididunt ut labore et dolore magna aliqua. Ut enim ad minim veniam, quis nostrud exercitation ullamco laboris nisi ut aliquip ex ea commodo consequat. Duis aute irure dolor in reprehenderit in voluptate velit esse cillum dolore eu fugiat nulla pariatur. Excepteur sint occaecat cupidatat non proident, sunt in culpa qui officia deserunt mollit anim id est laborum.

\begin{figure}[t]
    \centering
    \includegraphics[width=\linewidth]{placeholder.jpg}
    \caption{Single-wide figure.}
    \label{fig:result1}
\end{figure}

Lorem ipsum dolor sit amet, consectetur adipiscing elit, sed do eiusmod tempor incididunt ut labore et dolore magna aliqua. Ut enim ad minim veniam, quis nostrud exercitation ullamco laboris nisi ut aliquip ex ea commodo consequat. Duis aute irure dolor in reprehenderit in voluptate velit esse cillum dolore eu fugiat nulla pariatur. Excepteur sint occaecat cupidatat non proident, sunt in culpa qui officia deserunt mollit anim id est laborum.

\begin{figure*}[t]
    \centering
    \includegraphics[width=0.4\linewidth]{placeholder.jpg}
    \includegraphics[width=0.4\linewidth]{placeholder.jpg}
    \caption{Double-wide figure. \emph{Left:} My result was spectacular. \emph{Right:} Curious.}
    \label{fig:result2}
\end{figure*}

Lorem ipsum dolor sit amet, consectetur adipiscing elit, sed do eiusmod tempor incididunt ut labore et dolore magna aliqua. Ut enim ad minim veniam, quis nostrud exercitation ullamco laboris nisi ut aliquip ex ea commodo consequat. Duis aute irure dolor in reprehenderit in voluptate velit esse cillum dolore eu fugiat nulla pariatur. Excepteur sint occaecat cupidatat non proident, sunt in culpa qui officia deserunt mollit anim id est laborum.

%-------------------------------------------------------------------------
\subsection{Technical Discussion}

What about your method raises interesting questions? Are there any trade-offs? What is the right way to think about the changes that you made?

Lorem ipsum dolor sit amet, consectetur adipiscing elit, sed do eiusmod tempor incididunt ut labore et dolore magna aliqua. Ut enim ad minim veniam, quis nostrud exercitation ullamco laboris nisi ut aliquip ex ea commodo consequat. Duis aute irure dolor in reprehenderit in voluptate velit esse cillum dolore eu fugiat nulla pariatur. Excepteur sint occaecat cupidatat non proident, sunt in culpa qui officia deserunt mollit anim id est laborum.

%------------------------------------------------------------------------
\section{Conclusion}

What you did, why it matters, what the impact is going forward.

{\small
\bibliographystyle{plain}
\bibliography{ProjectFinal_ProjectReportTemplate}
}

\section*{Appendix}

\subsection*{Team contributions}

Please describe in one paragraph per team member what each of you contributed to the project.
\begin{description}
\item[Person 1 (put your real names)] Lorem ipsum dolor sit amet, consectetur adipiscing elit, sed do eiusmod tempor incididunt ut labore et dolore magna aliqua. Ut enim ad minim veniam, quis nostrud exercitation ullamco laboris nisi ut aliquip ex ea commodo consequat. Duis aute irure dolor in reprehenderit in voluptate velit esse cillum dolore eu fugiat nulla pariatur. 
\item[Person 2 (put your real names)] Lorem ipsum dolor sit amet, consectetur adipiscing elit, sed do eiusmod tempor incididunt ut labore et dolore magna aliqua. Ut enim ad minim veniam, quis nostrud exercitation ullamco laboris nisi ut aliquip ex ea commodo consequat. Duis aute irure dolor in reprehenderit in voluptate velit esse cillum dolore eu fugiat nulla pariatur.
\item [Person 3 (put your real names)] Lorem ipsum dolor sit amet, consectetur adipiscing elit, sed do eiusmod tempor incididunt ut labore et dolore magna aliqua. Ut enim ad minim veniam, quis nostrud exercitation ullamco laboris nisi ut aliquip ex ea commodo consequat. Duis aute irure dolor in reprehenderit in voluptate velit esse cillum dolore eu fugiat nulla pariatur. 
\item [Person 4 (put your real names)] Lorem ipsum dolor sit amet, consectetur adipiscing elit, sed do eiusmod tempor incididunt ut labore et dolore magna aliqua. Ut enim ad minim veniam, quis nostrud exercitation ullamco laboris nisi ut aliquip ex ea commodo consequat. Duis aute irure dolor in reprehenderit in voluptate velit esse cillum dolore eu fugiat nulla pariatur.
\end{description}

\end{document}
